% Author: Alfredo Sánchez Alberca (asalber@ceu.es)
\begin{tikzpicture}
    \begin{axis}[
        axis lines=middle,
        enlargelimits=true,
        %grid=major,
        xlabel=$x$,
        ylabel=$y$,
        xmin=0, xmax=7,
        ymin=0, ymax=1.2,
        xtick={1,2,3,4,5,6},
        ytick=\empty,
        yticklabels={},
        clip=false,
        %axis on top,
        domain=1:7,
        samples=100,
        %xlabel near ticks,
        %ylabel near ticks
        myblack
    ]
        % Plot the function 1/x
        \addplot[myblue, thick] {1/x} node[pos=0.85, anchor=south west] {$y=f(x)$};
    
        % Add Riemann rectangles
        \addplot[ybar interval, mark=no, fill=myblue, opacity=0.5] plot coordinates {
            (1, 1)
            (2, 1/2)
            (3, 1/3)
            (4, 1/4)
            (5, 1/5)
            (6, 1/6)
            (6, 0)  % This is to complete the last rectangle
        };
        \node at (axis cs:1.5,0.1) {$a_1$};
        \node at (axis cs:2.5,0.1) {$a_2$};
        \node at (axis cs:3.5,0.1) {$a_3$};
        \node at (axis cs:4.5,0.1) {$a_4$};
        \node at (axis cs:5.5,0.1) {$a_5$};
        \node at (axis cs:6.5,0.1) {$\cdots$};
        \node[myblue] at (axis cs:4.5,0.7) {$\displaystyle \sum_{n=1}^\infty a_n$};
    
    \end{axis}
    \end{tikzpicture}