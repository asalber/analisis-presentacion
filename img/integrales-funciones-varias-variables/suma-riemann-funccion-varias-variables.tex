% Author: Alfredo Sánchez Alberca (asalber@ceu.es)
\begin{tikzpicture}[scale=1.5,
    x=(215:2em/sqrt 2), y=(325:2em/sqrt 2), z=(90:1.5em),
    declare function={f(\x,\y)=((\x-3)^2+(-\y+2)^3)/8+4;},  
    thick, line join=round]
  \draw [-stealth, myblack] (0,0,0) -- (6,0,0) node [below left] {$x$};
  \draw [-stealth, myblack] (0,0,0) -- (0,6,0) node [below right] {$y$};
  \draw [-stealth, myblack] (0,0,0) -- (0,0,5) node [right] {$z$};
  \foreach \x in {1,...,4}
    \foreach \y [evaluate={\j=\x+.5; \i=\y+.5; \k=f(\j,\i);}] in {1,...,4}{
      \path [fill=myblue!50, draw=white] (\x, \y+1, 0) -- (\x+1, \y+1, 0) -- 
        (\x+1, \y+1, \k) -- (\x, \y+1, \k) -- cycle;
      \path [fill=myblue!25, draw=white] (\x+1, \y, 0) -- (\x+1, \y+1, 0) -- 
        (\x+1, \y+1, \k) -- (\x+1, \y, \k) -- cycle;
      \path [fill=myblue!10, draw=white] (\x, \y, \k)  -- (\x+1, \y, \k) -- 
        (\x+1, \y+1, \k) -- (\x, \y+1, \k) -- cycle;
    }

   \foreach \x in {1,...,4}
     \foreach \y in {1,...,4}{
   \draw [myblue, fill=myblue, fill opacity=0.125, 
      domain=0:1, samples=10, variable=\t] 
      plot (\x+\t, \y, {f(\x+\t,\y)}) -- 
      plot (\x+1, \y+\t, {f(\x+1,\y+\t)}) -- 
      plot (\x+1-\t, \y+1, {f(\x+1-\t,\y+1)}) --
      plot (\x, \y+1-\t, {f(\x,\y+1-\t)}) -- cycle;
    }
    \node[myblue] at (0, 4, 4) {$f(x,y)$};
  \end{tikzpicture}